\chapter{Introduction and Background}
\label{chap:introduction}

  \section{The data recorded at Beamline 4A}
      The raw data recorded at the magnetism reflectometer stores each detected neutron as one event, which includes information on
      the position on the detector, the relative time passed after the neutron pulse was created at the target, the absolute time of the
      according pulse and some instrument flags as the flipper ON/OFF state.
      After one run is finished this data is translated to the NeXus (.nxs) file format (which is based on the HDF5 standard) as two separate files,
      one with event information (flexible) and one with 3D histograms in X,Y and time of flight coordinates (fast).
      
      While the X and Y position can be used in conjunction with the instrument motor positions to gain information about the scattering
      angle the time of flight (ToF) together with the moderator to detector distance allows to deduce the neutron wavelength.
      Combining these three degrees of freedom allows to transform the coordinates into reciprocal space coordinates (for reflectivity
      only the \Qz coordinate is relevant.
      
      In order to derive the correct \Qz dependent intensities a normalization to the incident beam is necessary. 
      For this the direct beam intensity of the same wavelength band and with the same instrumental setup is measured
      prior to the actual data. This measurement is later used as reference when calculating the reflectivity.
      When different incident angles are measured (as is most often the case), each incident angle takes a different part out of
      the direct beam and generally won't have the same scaling factor as the earlier measurement. For the extracted reflectivity
      these different runs need to be combined.
  
  \section{What does QuickNXS data reduction do?}
      QuickNXS is a comprehensive tool to carry out the operations described above starting from the NeXus files and export the
      data in a form usable for plotting and fit with reflectivity modeling software.
      The program includes real time plots of several projections of the raw file data and previews of the exported data.
      Several automated algorithms aid the process to improve the learning curve of non expert users and speed up
      the extraction process of normal datasets while keeping the flexibility to apply special treatments for more
      complicated cases.
      \textbf{For a quick start reflectivity extraction guide see section \ref{sec:quick_start}.}
